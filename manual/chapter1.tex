\chapter{Wie erstelle ich eine neue Grammatik?}

Wir wollen das jetzt einmal an Hand folgenden Beispiels aus den
Vorlesungsunterlagen durchgehen:\vspace{10pt}



\begin{tabular}{lcr}
G = ($\Sigma, N, S, P ) mit $\\
	$\Sigma = \{0, 1, x, y, -, +, *, (, )\}$\\
	$N = {E}$\\
	$S=E$\\ 
	$P = \{E \to 0,\ E \to 1,\ E \to x,\ E \to y,\ E \to (-E),$\\
	$\ \ \ \ \ \ \ \ E \to (E + E),\ E \to (E - E),\ E \to (E $*$ E)\}$\\
\end{tabular}

\section{Neue Datei anlegen}
 
Am Anfang müssen wir dazu erst einmal den "`Neu\ldots"' Dialog öffnen. Dort
wählen wir aus, dass wir eine Grammatik erstellen wollen und klicken auf
"`Weiter"'.\vspace{10pt}

Im nächsten Dialog werden wir gefragt von welchem Typ unsere neue Grammatik
sein soll. In unserem Beispiel handel es sich um eine "`Kontextfreie Grammatik"',
welches wir auswählen und wieder mit "`Weiter"' bestätigen.\vspace{10pt}

Als nächstes müssen wir die benötigten Nichtterminalzeichen (N),
Terminalzeichen ($\Sigma$) und das Startzeichen (S) angeben.  Als Zeichen
können alle Symbole werwendet werden, die nicht zur Syntax der Eingabe gehören.
Das bedeutet "`$,$"', "`$\{$"`,"'$\}$"' und "`$"$"' können nicht verwendet
werden.\\

In dem Dialog sind an allen Stellen schon vordefinierte Werte eingetragen.
Dabei handelt es sich um die Vorgaben die man als Standardwerte in den Einstellung eingetragen hat.

Wie man diese Standardwerte ändert kann man in Kapitel $4$ nachlesen. Wir tragen
jetzt als erstes die von uns benötigten Nichtterminalzeichen ein. Dabei handelt
es sich bei uns lediglich um das Symbol "`E"'. Also schreiben wir in das Feld
"`\{E\}"'.\\

Als nächstes wollen wir das Startzeichen festlegen. Das ist bei uns das "`E"'
was wir in dem Feld für das Startzeichen schreiben.\\
Zum Schluss müssen wir noch die Terminalzeichen festlegen. Dazu tragen
wir in das entsprechende Feld "`\{0, 1, x, y, -, +, *, (, )\}"' ein.\\
Damit haben wir alle benötigten Information eingegeben und können die neue
Grammatik durch klick auf "`Fertig"' erstellen.\\

Alle Einstellungen die wir gerade für die neue Datei getroffen haben lassen
sich über "`Dokument editieren"'

\section{Anlegen von Produktionen}

Um eine neue Produktion anzulegen können wir den entsprechenden Button in der
Toolbar verwenden, oder über den Kontextmenüeintrag.\\

Im Dialog für neue Produktionen müssen wir zunächst einmal angeben, für welches
Nichtterminalzeichen wir eine Produktion anlegen möchten. Dieses können wir aus
einer Liste der verfügbaren Zeichen auswählen. Da wir in unserem Beispiel nur
ein Nichtterminalzeichen haben, ist schon das richtige ausgewählt, und wir
können diesen Schritt überspringen.\\

Jetzt müssen wir noch das Produktions Wort angeben. Als Hilfestellung wird in
dem Dialog angezeigt welche Nichtterminalzeichen und Terminalzeichen uns zur
Verfügung stehen. Wir fangen an mit der Produktion "`$E \to 0$"'. Also geben
wir als Produktions Wort "`0"' ein. Als weitere Hilfestellung wird im unteren
Dialog die resultierende Produktion im ganzen angezeigt. Wir bestätigen den
Dialog noch mit "`Ok"' und die neue Produktion erscheint in unserer Liste.\\
Wir können die Produktion selbstverständlich auch editieren und wieder löschen.
Beide Funktionen sind über das Kontextmenü für die ausgewählte Produktion
verfügbar.\\

Diesen Vorgang wiederholen wir jetzt für unsere komplette Menge "`P"'. Wenn wir
alle Produktionen angelegt haben sind wir auch mit dem anlegen der Grammatik
fertig. Es besteht jetzt noch die Möglichkeit die Grammatik zu validieren, um
zu sehen, ob man beim Erstellen irgendwelche Fehler gemacht hat. Diese Funktion
erreicht man über das Kontextmenü oder über den Menüpunkt "`Ausführen"'.

\section{Was fange ich mit einer fertigen Grammatik an?}

Wir haben jetzt eine Grammatik angelegt und stellen und jetzt vielleicht die
Frage was wir damit eigentlich anfangen können. Es bestehen zur Zeit zwei
Verwendungsmöglichkeiten für eine Grammatik.\\

Zunächst einmal kann man die Grammatik als Vorlage für eine neue Grammatik
nutzen. Das ist zum Beispiel sinnvoll wenn wir eine kontextfreie Grammatik
haben, und jetzt eine reguläre Grammatik mit den gleichen Produktionen
erstellen wollen. Dazu öffnet man den Menüpunkt "`Ausführen"' und wählt unter
"`Vorlage für\ldots"' den gewünschten Typ der neuen Datei aus.\\

Die andere Verwendungsmöglichkeit ist, sich aus der Grammatik einen
entsprechenden Automaten generieren zu lassen. Man kann sich für eine reguläre
Grammatik den NDEA, und für eine kontextfreie Grammatik den Kellerautomat
erzeugen lassen.\\

Es ist zu beachten, dass beim Umwandeln einer Grammatik keine
Validierungsfehler vorhanden sein dürfen. Bei der Funktion "`Vorlage
für\ldots"' spielen Fehler allerdings keine Rolle.




