\chapter{Wie erstelle ich einen Automaten?}

Wir wollen uns ahand des folgenden Beispiels anschauen wie ein Automat erstellt
wird:\vspace{10pt}

\section{Neue Automat Datei anlegen}

Am Anfang müssen wir dazu erst einmal den "`Neu\ldots"' Dialog öffnen. Dort
wählen wir aus, dass wir einen Automaten erstellen wollen und klicken
auf "`Weiter"'.\vspace{10pt}

Im nächsten Dialog werden wir gefragt von welchem Typ unser neuer
Automat sein soll. In unserem Beispiel handelt es sich um eine
"`$\epsilon$-NDEA"', welches wir auswählen und wieder mit "`Weiter"'
bestätigen.\vspace{10pt}

Als nächstes müssen wir das gewünschte Alphabet für den neuen Automaten
angeben. Im Alphabet können alle Symbole werwendet werden, die nicht zur Syntax
der Eingabe gehören. Das bedeutet \Symbol{,}, \Symbol{\{}, \Symbol{\}} und \Symbol{"} können nicht verwendet
werden.\vspace{10pt}

In dem Dialog sind an allen Stellen schon vordefinierte Werte eingetragen.
Dabei handelt es sich um die Vorgaben die man als Standardwerte in den
Einstellung eingetragen hat. Wie man diese Standardwerte ändert kann man in Kapitel $4$ nachlesen.\vspace{10pt}

Wir müssen noch die von uns benötigten Symbole angeben. Dazu tragen wir in das
Feld Alphabet \{\Symbol{0},\ \Symbol{1},\ \Symbol{2},\ \Symbol{3},\
\Symbol{4},\ \Symbol{5},\ \Symbol{6},\ \Symbol{7},\ \Symbol{8},\ \Symbol{9},\
\Symbol{-}\} ein. Damit haben wir alle benötigten Information eingegeben und
können den neuen Automaten durch klick auf "`Fertig"' erstellen.\vspace{10pt}
\vspace{10pt} 

Das für den Automaten angegebene Alphabet lässt sich jeder Zeit über "`Dokument
editieren"' nächträglich ändern.

\section{Zustände und Übergänge anlegen}

Um einen Automaten zu bearbeiten müssen wir zuerst den entsprechenden Modus
auswählen. Folgende Modi sind in der Toolbar verfügbar:

\begin{itemize}
  \item Maus Modus
  \item Neuer Zustand
  \item Neuer Übergang
  \item Neuer Start Zustand
  \item Neuer Finaler Zustand
\end{itemize}

Da wir zu Beginn den Startzustand anlegen wollen wählen wir "`Neuer Finaler
Zustand"' und klicken auf eine freie Fläche um den Zustand dort zu erstellen.
Wenn wir den Zustand bearbeiten wollen um den Namen zu ändern, oder ob es sich
um einen akzeptierenden oder Start Zustand handelt müssen wir in den Maus Modus
wechseln. Den Konfigurationsdialog erreicht man dann über Doppelklick oder das
Kontextmenü. Über das Kontextmenü lässt sich ein Zustand auch wieder
löschen.\vspace{10pt}

Auf diese Weise lassen sich alle Zustände anlegen die für den Beispiel
Automaten benötigt werden. Nachdem wir alle Zustände angelegt haben, fehlen
noch die Übergänge zwischen den Zuständen. Daher wechseln wir in den "`Neuer
Übergang"' Modus. Um jetzt einen neuen Übergang anzulegen klicken wir auf einen
Zustand und ziehen, bei gedrückter Maustaste, die Maus auf einen anderen
Zustand. Beim loslassen öffnet sich der Konfigurationsdialog für Übergänge. In
diesem Dialog ziehen wir jetzt die Symbole die in dem Übergang enthalten
sein sollen aus der linken Liste mit allen Symbolen in die Rechte Liste, welche
die Übergangsmenge representiert. Als Hilfestellung wird im unteren Dialog der
entstehende Übergang angezeigt. Nach bestätigen durch "`Ok"' wird der
Übergang angelegt.\vspace{10pt}

Die Übergänge lassen sich auf die gleiche Weise bearbeiten und löschen wie
es bei den Zuständen möglich ist. Es gibt allerdings eine Besonderheit beim
Anlegen von Übergängen. Wenn man die Maus nicht über einem Zustand, sondern
über einem leeren Bereich loslässt, wird beim Anlegen des Übergangs auch ein
neuer Zustand angelegt.\vspace{10pt}

Auf diese Weise legen wir alle Übergänge an, welche für den Beispielautomaten
benötigt werden. Wenn wir damit fertig sind, können wir den Automaten noch
validieren, um zu sehen ob uns beim Anlegen der Zustände und Übergänge
irgendwelche Fehler unterlaufen sind. Diese Funktion
erreicht man über das Kontextmenü oder über den Menüpunkt "`Ausführen"'.

\section{Was fange ich mit einem fertigen Automat an?}

Wenn man einen Automaten erstellt hat gibt es folgende
Ver\-wen\-dungs\-möglich\-keiten:

\begin{itemize}
  \item Vorlage für\ldots
  \item Wortnavigation
  \item Minimieren
  \item Umwandeln in\ldots
  \item Umwandeln in\ldots (Potenzmenge)
\end{itemize}

\subsection{Vorlage für\ldots}
  
  Ein Automat kann als Vorlage für einen neuen Automaten benutzt werden. Dies
  sollte nicht mit "`Umwandeln in\ldots"' verwechselt werden. Es wird eine neue
  Datei angelegt, die alle Zustände und Übergänge des Ausgangsautomaten hat.
  Allerdings kann man den Automatentypen neu festlegen.
  
\subsection{Wortnavigation}
  
  Wir können für einen Automaten eine Wortnavigation starten. Das bedeutet,
  dass wir ein Wort eingeben können, welches wir den Automaten verarbeiten
  lassen wollen, und dann Symbolweise vor und zurück navigieren können. Dabei
  werden die aktuell aktiven Zustände und Übergänge farblich hervorgehoben. Am
  Ende können wir dann sehen, ob der Automat das von uns gewählte Wort
  akzeptiert oder nicht.\vspace{10pt}
  
  Während man sich im Wortnavitionsmodus
  befindet lässt sich der Automat nicht weiter bearbeiten. Das bedeutet
  wir können Zustände und Übergänge weder anlegegen noch löschen
  oder bearbeiten. Dies ist erst nach verlassen dieses Moudus wieder
  möglich.\vspace{10pt}
  
  Zur Wortnavigation gelangt man über den Menüpunkt "`Ausführen"' und "`Wort
  eingeben"', oder direkt über den Togglebutton in der Toolbar. In dem
  Feld "`Wort"' können wir jetzt ein beliebiges Wort eingeben. Als
  Hilfestellung sehen wir rechts neben dem Eingabefeld das aktuelle Alphabet
  mit den gültigen Symbolen.\vspace{10pt}
  
  Nachdem wir das Wort eingegeben haben startet man die Navigation über
  "`Start"' ind der Toolbar oder im Kontextmenü. Jetzt kann man mit "`Schritt
  vor"' und "`Schritt zurück durch das Wort navigieren. Es existiert auch noch
  ein automatischer Modus. Dabei wird nach einer kurzen Verzögerung das nächste
  Symbol gelesen. Dieser Modus lässt sich durch den Togglebutton "`Automatische
  Schritte"' aktivieren und deaktivieren.\vspace{10pt}
  
  Um die Wortnavigation zu beenden klickt man einfach wieder auf "`Wort
  eingeben"' in der Toolbar oder man wählt im Menü "`Ausführen"' den Punkt
  "`Automat bearbeiten"'. Jetzt befinden wir uns wieder im normalen Modus und
  wir können den Automaten wieder normal bearbeiten.
   
\subsection{Minimieren}
  
  TODO BM
  
\subsection{Umwandeln in\ldots}
  
  TODO CF
  
\subsection{Umwandeln in\ldots (Potenzmenge)}
  
  TODO CF
  
  
