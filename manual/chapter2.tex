\chapter{Wie erstelle ich einen Automaten?}

Wir wollen uns ahand des folgenden Beispiels anschauen wie ein Automat erstellt
wird:\vspace{10pt}

\section{Neue Automat Datei anlegen}

Am Anfang müssen wir dazu erst einmal den "`Neu\ldots"' Dialog öffnen. Dort
wählen wir aus, dass wir einen Automaten erstellen wollen und klicken
auf "`Weiter"'.\vspace{10pt}

Im nächsten Dialog werden wir gefragt von welchem Typ unser neuer
Automat sein soll. In unserem Beispiel handelt es sich um eine
"`$\epsilon$-NDEA"', welches wir auswählen und wieder mit "`Weiter"'
bestätigen.\vspace{10pt}

Als nächstes müssen wir das gewünschte Alphabet für den neuen Automaten
angeben. Im Alphabet können alle Symbole werwendet werden, die nicht zur Syntax
der Eingabe gehören. Das bedeutet "`$,$"', "`$\{$"`,"'$\}$"' und "`$"$"' können nicht verwendet
werden.\vspace{10pt}

In dem Dialog sind an allen Stellen schon vordefinierte Werte eingetragen.
Dabei handelt es sich um die Vorgaben die man als Standardwerte in den Einstellung eingetragen hat.

Wie man diese Standardwerte ändert kann man in Kapitel $4$ nachlesen. Wir tragen
jetzt als erstes die von uns benötigten Nichtterminalzeichen ein. Dabei handelt
es sich bei uns lediglich um das Symbol "`E"'. Also schreiben wir in das Feld
"`\{E\}"'.\vspace{10pt}

Alle Einstellungen die wir gerade im letzen Dialog für die neue Datei
getroffen haben lassen sich über "`Dokument editieren"'

\section{Zustände und Übergänge anlegen}

Um einen Automaten zu bearbeiten müssen wir zuerst den entsprechenden Modus
auswählen. Folgende Modi sind in der Toolbar verfügbar:

\begin{itemize}
  \item Maus Modus
  \item Neuer Zustand
  \item Neuer Übergang
  \item Neuer Start Zustand
  \item Neuer Finaler Zustand
\end{itemize}

Da wir zu Beginn den Startzustand anlegen wollen wählen wir "`Neuer Finaler
Zustand"' und klicken auf eine freie Fläche um den Zustand dort zu erstellen.
Wenn wir den Zustand bearbeiten wollen um den Namen zu ändern, oder ob es sich
um einen akzeptierenden oder Start Zustand handelt müssen wir in den Maus Modus
wechseln. Den Konfigurationsdialog erreicht man dann über Doppelklick oder das
Kontextmenü. Über das Kontextmenü lässt sich ein Zustand auch wieder
löschen.\vspace{10pt}

Auf diese Weise lassen sich alle Zustände anlegen die für den Beispiel
Automaten benötigt werden. Nachdem wir alle Zustände angelegt haben, fehlen
noch die Übergänge zwischen den Zuständen. Daher wechseln wir in den "`Neuer
Übergang"' Modus. Um jetzt einen neuen Übergang anzulegen klicken wir auf einen
Zustand und ziehen, bei gedrückter Maustaste, die Maus auf einen anderen
Zustand. Beim loslassen öffnet sich der Konfigurationsdialog für Übergänge. In
diesem Dialog ziehen wir jetzt die Symbole die in dem Übergang enthalten
sein sollen aus der linken Liste mit allen Symbolen in die Rechte Liste, welche
die Übergangsmenge representiert. Als Hilfestellung wird im unteren Dialog der
entstehende Übergang angezeigt. Nach bestätigen durch "`Ok"' wird der
Übergang angelegt.\vspace{10pt}

Die Übergänge lassen sich auf die gleiche Weise bearbeiten und löschen wie
es bei den Zuständen möglich ist. Es gibt allerdings eine Besonderheit beim
Anlegen von Übergängen. Wenn man die Maus nicht über einem Zustand, sondern
über einem leeren Bereich loslässt, wird beim Anlegen des Übergangs auch ein
neuer Zustand angelegt.\vspace{10pt}

Auf diese Weise legen wir alle Übergänge an, welche für den Beispielautomaten
benötigt werden. Wenn wir damit fertig sind, können wir den Automaten noch
validieren, um zu sehen ob uns beim Anlegen der Zustände und Übergänge
irgendwelche Fehler unterlaufen sind. Diese Funktion
erreicht man über das Kontextmenü oder über den Menüpunkt "`Ausführen"'.
