\chapter{Sonstiges}

\section{Zweite Ansicht}

TODO CF

\section{Austausch von Dateien}

Das \gtitool bietet die Möglichkeit geöffnete Dateien auszutauschen und Dateien
von anderen \gtitool Benutzern zu empfangen. Zu erreichen ist der Austausch über
den Hauptmenüeintrag "`Extras"'. Die Dateien werden verschlüsselt
übertragen.\vspace{10pt}

Wenn eine Datei verschickt werden soll, muss diese als erstes selektiert werden.
Anschließend muss der Austausch Dialog geöffnet werden. Dieser bietet die
Möglichkeit zwischen "`Senden"' und "`Empfangen"' zu wählen. Um die Datei zu
verschicken muss also "`Senden"' ausgewählt werden, außerdem muss der "`Host"'
angegeben werden, auf dem ein \gtitool empfangsbereit ist, zusätzlich ist der
"`Port"' anzugeben, standardmäßig ist Port "`64528"' eingestellt. Der Port muss
also nur geändert werden, wenn der Empfänger der Datei ihn ebenfalls geändert
hat. Beim Senden kann zusätzlich ein Kommentar vergeben werden, den der Empfänger
angezeigt bekommt, wenn die Datei empfangen wurde, der aber sonst keinen
Einfluss hat. Wenn alle Werte eingestellt sind, kann die Datei mit
"`Ausführen"' verschickt werden.\vspace{10pt}

Das Empfangen einer Datei funktioniert sehr einfach, es muss nur der Port
eingestellt werden, auf dem der Server auf eingehende Dateien horchen soll. Per
"`Ausführen"' horcht der Server auf dem eingestellten Port auf eintreffende
Dateien. Per "`Abbrechen"' kann der Server wieder gestoppt werden, sollte keine
Datei empfangen werden. Wird eine Datei empfangen, wird diese automatisch
geöffnet und der Server beendet. Der Austausch Dialog wird nicht geschlossen, so
dass weitere Dateien empfangen oder verschickt werden können. Wird eine Datei
empfangen, kann der vom Sender eingestellte Kommentar im Nachrichtenfenster
eingesehen werden.

\section{Bild exportieren}

Im Menüpunkt "`Extras"' besteht die Möglichkeit den Automaten Graphen in ein
Bild zu exportieren. Der Eintrag ist nur aktiv, wenn eine Automaten Datei
selektiert ist. Wird der Eintrag "`Bild exportieren"' angeklickt, öffnet sich
ein Speichern Dialog mit dem das Bild gespeichert werden kann. Damit das
Ergebnis nicht zu viele freie Bereiche enthält, wird nur der vom Automaten
benutzte Bereich exportiert, zusätzlich zu einem 20 Pixel breiten Rand.