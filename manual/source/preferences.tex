%%
%% $Id$
%%
%% Copyright (c) 2007-2008 Christian Fehler
%% Copyright (c) 2007-2008 Benjamin Mies
%%


\chapter{Einstellungen}\label{Preferences}

Die Oberfläche des \gtitools kann den eigenen Bedürfnissen angepasst werden.
Dazu kann man die Einstellungen im Menüpunkt "`Bearbeiten"' öffnen. Alle
Einstellungen werden erst bei Klick auf "`Übernehmen"' oder "`OK"' übernommen.
Per Klick auf "`Standard"' kann man die Einstellungen auf die Standardwerte
zurücksetzen.


\section{Oberfläche anpassen}

Auf dem Tab "`Allgemein"' kann die Sprache eingestellt werden, zur Auswahl
stehen im Moment Deutsch und Englisch. Ist "`Default"' ausgewählt, wird
die Systemsprache verwendet, sollte diese Deutsch oder Englisch sein. Ist dies
nicht der Fall, wird Englisch verwendet.\vspace{10pt}

\gtitool verwenden standardmäßig "`TinyLaF"' als "`Look \& Feel"', es kann aber
auch jedes andere auf dem System zur Verfügung stehende Look \& Feel benutzt
werden.\vspace{10pt}

Die Option "`Wort Modus"' bezieht sich auf die Darstellung des Wortes während
der Wort-Navigation. Ist "`Linksbündig"' ausgewählt, wird das Wort normal
linksbündig dargestellt. Während der Navigation werden schon gelesene Symbole
hervorgehoben. Ist dagegen "`Rechtsbündig"' ausgewählt, wird das Wort genau wie
der Keller rechtsbündig dargestellt. Eine weitere Änderung ist, dass die
gelesenen Symbole nicht hervorgehoben, sondern ausgeblendet werden.\vspace{10pt}

Die Ein\-stell\-ungs\-möglich\-keit "`Zoom Faktor"' bezieht sich auf den Graphen
des Automaten, es handelt sich dabei um die für jede Datei benutzte
Voreinstellung. Der Zoom Faktor kann aber auch für jede Datei einzeln eingestellt
werden.\vspace{10pt}

\newpage
Auf dem Tab "`Ansicht"' kann die Einstellung "`Auto Step"' angepasst werden. Im
\gtitool kann an verschiedenen Stellen, zum Beispiel bei der Wort-Navigation,
diese Navigation auch automatisch ablaufen. Der hier eingestellte Wert wird dabei
als Zeit zwischen zwei Schritten verwendet.\vspace{10pt}

\begin{figure}[h]
\begin{center}
\includegraphics[width=8cm]{../images/preferences.png}
\caption{Einstellungen}
\end{center}
\end{figure}

Der Tab "`Farben"' bietet die Möglichkeit, die verwendeten Farben seinen
Be\-dürf\-nis\-sen anzupassen. Zur Auswahl stehen sowohl die beim Automaten, wie
auch die bei der Grammatik verwendeten Farben, aber auch einige andere. Der
Tooltip, der auf jeder Farbe angezeigt wird, gibt genauere Informationen
darüber, wo die Farbe verwendet wird.


\section{Automaten Einstellungen anpassen}

Für einen Automaten gibt es verschiedene Einstellungsmöglichkeiten, so kann
eingestellt werden, wie Übergänge angelegt werden können und wie sich die
Auswahl der aktiven Buttons zum Anlegen von Übergängen und Zuständen verhält.
Das Anzeigen der Komponenten, die nur bei Kellerautomaten einen richtigen Sinn
ergeben, kann angepasst werden und es kann eine Vorauswahl des verwendeten
Alphabetes getroffen werden.\vspace{10pt}

Die Einstellung "`Übergang"' bietet zwei Möglichkeiten. Wird "`Maus-Drag Modus"'
ausgewählt, kann ein Übergang zwischen zwei Zuständen per "`Drag and Drop"'
angelegt werden. Dazu muss der Drag auf dem Anfangszustand beginnen, der Drop auf
dem Endzustand oder auf einer freien Stelle, wobei dann ein neuer Zustand
angelegt wird, erfolgen. Die zweite Möglichkeit ist die Einstellung "`Maus-Klick
Modus"'. Ist diese ausgewählt, muss auf dem Anfangszustand ein Mausklick
erfolgen, anschließend kann auf dem Endzustand oder auf einer freien Stelle
wieder mit der Maus geklickt werden, um den Übergang anzulegen.\vspace{10pt}

Der Punkt "`Maus Auswahl"' bietet die Möglichkeit, auszuwählen, welcher Button
nach dem Anlegen eines Zustandes oder eines Überganges aktiviert wird. Ist "`Ohne
Rücksprung zur Maus"' ausgewählt, bleibt der zuletzt aktive Button unverändert.
In diesem Modus können also mehrere Übergänge nacheinander angelegt werden, ohne,
dass jedesmal wieder der Button "`Neuer Übergang"' aktiviert werden muss. Wenn
"`Mit Rücksprung zur Maus"' ausgewählt ist, wird nach jedem Anlegen wieder der
Maus Button aktiviert. Dieser Modus ist besser dazu geeignet, wenn nur einzelne
Zustände oder Übergänge angelegt werden müssen, da nach dem Anlegen direkt wieder
Komponenten im Graphen ausgewählt werden können.\vspace{10pt}

Mit der Option "`Kellerautomat Modus"' kann eingestellt werden, welche Informationen
der Benutzer angezeigt bekommt, wenn er einen Automaten ohne Keller bearbeitet. Ist
"`Alle Komponenten anzeigen"' ausgewählt, werden auch auf einem Automaten ohne Keller
die Information zu diesem angezeigt. Dies betrifft die "`Keller Operationen Tabelle"',
das Eingeben der Wörter, die vom Keller gelesen, bzw. auf ihn geschrieben
werden im Übergangs Dialog und noch einige andere Komponenten. Ist "`Nur anzeigen, wenn
Kellerautomat ausgewählt"' aktiviert, werden diese Komponenten bei Automaten ohne
Keller ausgeblendet, bzw. deaktiviert.\vspace{10pt}

Im Tab "`Alphabet"' befindet sich die Möglichkeit, das Eingabe-Alphabet, sowie
das Keller-Alphabet zu bearbeiten. Bei den hier eingestellten Werten handelt es sich
um die Voreinstellung für neue Dateien. Beide Alphabete können auf jeder Datei
einzeln angepasst und gespeichert werden. Die Eingabe kann in geschweiften
Klammern erfolgen und muss auf jeden Fall mit Komma getrennt werden. Die
verwendeten Symbole dürfen nur ein Zeichen enthalten, oder müssen in
Anführungszeichen stehen. Somit wäre zum Beispiel \{\Symbol{0}, \Symbols{if}, 
\Symbols{then}\} eine gültige Eingabe. Mit dem Haken bei "`Keller Alphabet"' kann
man angeben, dass man ein eigenes Keller-Alphabet benutzen will. Ist der Haken
nicht gesetzt, wird als Keller-Alphabet das Eingabe-Alphabet verwendet. Die
Eingabe des Keller-Alphabets erfolgt auf gleiche Weise wie die des
Eingabe-Alphabets.


\section{Grammatik Einstellungen anpassen}

Im Tab "`Grammatik"' können die Voreinstellungen für Nichtterminalzeichen,
Terminalzeichen und für das verwendete Startsymbol, das aus der Menge der
Nichtterminalzeichen sein muss, eingestellt werden. Die Eingabe muss genauso wie
beim Alphabet erfolgen. Zu beachten ist, dass die Menge der Nichtterminalzeichen
und die Menge der Terminalzeichen disjunkt sein müssen. Auch hierbei handelt es
sich nur um die Voreinstellung die im "`Neu-Dialog"' verwendet wird.