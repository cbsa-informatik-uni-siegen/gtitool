\chapter{Wie erstelle ich einen regulären Ausdruck?}

Wir wollen uns dies anhand des folgenden Beispiels anschauen: \Symbol{(ab)*abb}

\section{Neue Regex Datei anlegen}

Zu Beginn müssen wir dazu, wie auch beim Automat und bei Grammatiken, den "`Neu\ldots"' Dialog öffnen. Zu finden ist dieser in der Toolbar oder im Menüeintrag "`Datei"'. Dort wählen wir "`Regulärer Ausdruck"' an und klicken auf "`Weiter"'.\vspace{10pt}

Danach müssen wir im nächsten Dialog nur noch ein Alphabet festlegen, auf welchem der Reguläre Ausdruck definiert ist. Hier ist es auch möglich eine Klasse von Zeichen anzugeben, zum Beispiel \Symbol{[a-z]}.\vspace{10pt}

In unserem Beispiel würde also ein Alphabet bestehend aus \Symbol{a} und \Symbol{b} genügen. Man kann dieses Alphabet auch im Nachhinein noch verändern über den Button "`Dokument editieren"'. Nachdem wir ein Alphabet festgelegt haben bestätigen wir dies mit "`Fertig"'.\vspace{10pt}

\section{Regluären Ausdruck eingeben}

Nun ist folgender Bildschirm zu sehen:\vspace{10pt}

Oben rechts ist das Eingabefeld für den Reglulären Ausdruck darunter findet man nochmal zur Orientierung das Alphabet, dass wir gerade eingegeben haben, verändert werden kann es aber nur über den Button "`Dokument editieren"'. Im Eingabefeld tragen wir nun unseren Beispielausdruck \Symbol{(ab)*abb} ein.\vspace{10pt}

Direkt bei der Eingabe wird im Graph-Fenster in der Mitte der Reguläre Ausdruck als Baum dargestellt.
