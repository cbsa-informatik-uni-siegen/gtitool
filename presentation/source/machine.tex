%%
%% $Id$
%%
%% Copyright (c) 2007-2008 Christian Fehler
%% Copyright (c) 2007-2008 Benjamin Mies
%%


%### removes texlipse warnings


\myslide{Automaten}
{
    \begin{itemgroup}{}
	\item Graphenansicht
	\item Tabellen
	\item Wort-Navigation
	\item Erreichbare Zustände
	\item Konvertierung
	\item Minimierung
	\end{itemgroup}

	\vfill{}
    
}

\myslide{Automaten - Graphenansicht}
{
    \begin{itemgroup}{}
	\item Orientierung an der Vorlesung
	\item Normale Zustände werden durch Kreis mit Namen in der Mitte dargestellt
	\begin{itemgroup}{}
		\item Startzustände haben zusätzlich einen mit Start beschrifteten Pfeil
		\item Akzeptierende Zustände haben zusätzlich einen doppelten Rahmen
	\end{itemgroup}
	\item Transitionen werden als Pfeil mit Übergangsmenge dargestellt
	\end{itemgroup}

	\vfill{}
}

\myslide{Automaten - Graphenansicht}
{
  \begin{center}
    \includegraphics[height=14cm]{../images/enfa_example.png}
  \end{center}
}

\myslide{Automaten - Tabellen}
{
TODOCF Bearbeiten von Tabellen
    \begin{itemgroup}{Überschrift 0}
	\item Item 0
	\item Item 1
	\item Item 2
	\end{itemgroup}
    
	\begin{itemgroup}{Überschrift 1}
	\item Item 0
	\item Item 1
	\item Item 2
	\end{itemgroup}
}

\myslide{Automaten - Wort-Navigation}
{
    \begin{itemgroup}{}
	\item Eingabe eines Wortes
	\item Zeichenweise Navigation durch das Wort
	\item Aktiven Zustände und Übergänge werden hervorgehoben
	\item Es wird angezeigt wie weit das Wort bereits gelesen wurde
	\end{itemgroup}

	\vfill{}
}

\myslide{Automaten - Wort-Navigation}
{
  \begin{center}
    \includegraphics[height=14cm]{../images/dfa_navigation.png}
  \end{center}
}

\myslide{Automaten - Wort-Navigation}
{
  TODOCF Wort-Navigation Kellerautomat \\
  
      \begin{itemgroup}{Überschrift 0}
	\item Item 0
	\item Item 1
	\item Item 2
	\end{itemgroup}
    
	\begin{itemgroup}{Überschrift 1}
	\item Item 0
	\item Item 1
	\item Item 2
	\end{itemgroup}
}

\myslide{Automaten - Erreichbare Zustände}
{
TODOCF \\
    \begin{itemgroup}{Überschrift 0}
	\item Item 0
	\item Item 1
	\item Item 2
	\end{itemgroup}
    
	\begin{itemgroup}{Überschrift 1}
	\item Item 0
	\item Item 1
	\item Item 2
	\end{itemgroup}
}

\myslide{Automaten - Konvertierung}
{
    \begin{itemgroup}{}
	\item Konvertierung von Grammatiken
		\begin{itemgroup}{}
    	\item Reguläre Grammatik konvertieren
    	\item Kontextfreie Grammatik konvertieren
    	\end{itemgroup}
	\item Konvertierung von Automaten
		\begin{itemgroup}{}
    	\item TODOCF
    	\end{itemgroup}
	\end{itemgroup}
  
	\vfill{}
}

\myslide{Automaten - Reguläre Grammatik konvertieren}
{
    \begin{itemgroup}{}
	\item Es wird jede einzelene Produktion betrachtet
	\item Für das Nichtterminalzeichen auf der linken Seite wird ein Zustand
	angelegt.
	\item Besteht die rechte Seite aus einem einzelnen Terminalzeichen
  		\begin{itemgroup}{}
    	\item Einen Übergang zu einem akzeptierenden Zustand
    	\item Terminalzeichen als Produktionsmenge
    	\end{itemgroup}
	\item Besteht die rechte Seite aus einem Terminalzeichen und einem Nichtterminalzeichen
  		\begin{itemgroup}{}
    	\item Einen Übergang zu einem Zustand der das Nichterminalzeichen repräsentiert
		\item Terminalzeichen als Produktionsmenge
		 \end{itemgroup}
    \end{itemgroup}
	\vfill{}
}

\myslide{Automaten - Kontextfreie Grammatik konvertieren}
{
    \begin{itemgroup}{}
	\item Anlegen eines Startzustands
	\item Anlegen eines akzeptierenden Zustands
	\item Übergang von Startzustand in akzeptierenden Zustand bei dem das 
	Startsymbol auf den Keller gelegt wird
	\item Für jede Produktion ein Übergang vom akzeptierenden in akzeptierenden
	Zustand
		\begin{itemgroup}{}
    	\item Rechte Seite der Produktion vom Keller entfernen
    	\item Durch linke Seite der Produktion auf den Keller schreiben
    	\end{itemgroup}
	\item Für jedes Terminalzeichen ein Übergang vom akzeptierenden in akzeptierenden
	Zustand
		\begin{itemgroup}{}
    	\item Das Terminalzeichen wird vom Eingabeband gelesen und vom Keller
    	entfernt \end{itemgroup}
	\end{itemgroup}
  
	\vfill{}
}

\myslide{Automaten - Konvertierung von Automaten}
{
TODOCF \\
    \begin{itemgroup}{Überschrift 0}
	\item Item 0
	\item Item 1
	\item Item 2
	\end{itemgroup}
    
	\begin{itemgroup}{Überschrift 1}
	\item Item 0
	\item Item 1
	\item Item 2
	\end{itemgroup}
  
	\vfill{}
}

\myslide{Automaten - Minimieren}
{
    \begin{itemgroup}{}
	\item Nicht erreichbare Zustände entfernen
	\item Einteilung in zwei Äquivalenzklassen
		\begin{itemgroup}{}
		\item Akzeptierende Zustände
		\item Nicht akzeptierende Zustände
		\end{itemgroup}
	\item Verfeinern der Äquivalenzklassen
	\item Jede Äquivalenzklasse repräsentiert einen Zustand in dem minimalen
	Automat
	\end{itemgroup}
	
	\vfill{}    
}

\myslide{Automaten - Minimieren}
{
  \begin{center}
    \includegraphics[height=14cm]{../images/minimize.png}
  \end{center}   
}


%### removes texlipse warnings
