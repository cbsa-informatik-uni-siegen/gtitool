%%
%% $Id$
%%
%% Copyright (c) 2007-2008 Christian Fehler
%% Copyright (c) 2007-2008 Benjamin Mies
%%


%### removes texlipse warnings


\myslide{Perspektiven}
{
    \begin{itemgroup}{}
	\item Graphische Komponenten
	\item Reguläre Ausdrücke
	\item \LaTeX-Export
	\item Erkannte Wörter ausgeben
	\item Eingabe von mehreren Wörtern
	\item Benutzerinteraktion erhöhen
	\item Grammatik-Erweiterungen
	\item Auto-Layout
	\end{itemgroup}

    \vfill{}
}


\myslide{Perspektiven - Graphische Komponenten}
{
    \begin{itemgroup}{}
	\item Graphische Darstellung verbessern
	\item Bessere Wartbarkeit
    \end{itemgroup}

    \begin{itemgroup}{Lösung}
	\item Implementierung von Zuständen und Übergängen
	\item Nicht nur linear verlaufende Übergänge
	\end{itemgroup}
    
    \vfill{}
}


\myslide{Perspektiven - Reguläre Ausdrücke}
{
    \begin{itemgroup}{}
	\item Reguläre Ausdrücke unterstützen
	\item Einsatz im Bereich der Vorlesung Compilerbau I
    \end{itemgroup}

    \begin{itemgroup}{Lösung}
	\item Eingabe von regulären Ausdrücken
	\item Konvertierung nach McNaughton-Yamada-Thompson
	\item Direkte Konvertierung in einen DEA
	\end{itemgroup}
    
    \vfill{}
}


\myslide{Perspektiven - \LaTeX-Export}
{
    \begin{itemgroup}{}
	\item Bessere Unterstützung von Dozenten
	\item Verwendung in Vorlesungsskripten
    \end{itemgroup}

    \begin{itemgroup}{Lösung}
	\item Auswahl einer geeigneten \LaTeX-Bibliothek
	\item Dynamische Erzeugung eines \LaTeX-Dokumentes
	\item Ähnlicher Funktionsumfang wie bei TPML
	\end{itemgroup}
    
    \vfill{}
}


\myslide{Perspektiven - Erkannte Wörter ausgeben}
{
    \begin{itemgroup}{}
	\item Überprüfung der richtigen Funktion
	\item Sprache eines Automaten erkennen 
    \end{itemgroup}

    \begin{itemgroup}{Lösung}
	\item Liste der erkannten Wörter ausgeben
	\item Kürzere Wörter zuerst prüfen
	\item Nach Möglichkeit nicht per Brute-Force-Methode
	\end{itemgroup}
    
    \vfill{}
}


\myslide{Perspektiven - Eingabe von mehreren Wörtern}
{
    \begin{itemgroup}{}
	\item Überprüfung der richtigen Funktion
	\item Wörter schneller überprüfbar
    \end{itemgroup}

    \begin{itemgroup}{Lösung}
	\item Eingabe mehrerer Wörter in einer Liste
	\item Direkte Überprüfung der Wörter
	\end{itemgroup}
    
    \vfill{}
}


\myslide{Perspektiven - Benutzerinteraktion erhöhen}
{
    \begin{itemgroup}{}
	\item Inhalte noch besser vermitteln
	\item Benutzer besser integrieren
    \end{itemgroup}

    \begin{itemgroup}{Lösung}
	\item Wort-Navigation
	\item Konvertierung
	\item Erreichbare Zustände
	\item Minimierung
	\end{itemgroup}
    
    \vfill{}
}


\myslide{Perspektiven - Grammatik-Erweiterungen}
{
    \begin{itemgroup}{}
	\item Linksrekursion
	\item Linksfaktorisierung
	\item Ableitungen für Grammatiken
    \end{itemgroup}
    
    \vfill{}
}

\myslide{Perspektiven - Linksrekursion}
{
    \begin{itemgroup}{}
	\item Probleme zum Beispiel bei Produktionen der Form $\NonterminalSymbol{E}
	\to \NonterminalSymbol{E} + \NonterminalSymbol{E}$
	\item Einsatz im Bereich der Vorlesung Compilerbau I
    \end{itemgroup}

    \begin{itemgroup}{Lösung}
	\item Erkennen und entfernen von direkter Linksrekursion
	\item Erkennen und entfernen von indirekter Linksrekursion
	\end{itemgroup}
    
    \vfill{}
}

\myslide{Perspektiven - Linksfaktorisierung}
{
    \begin{itemgroup}{}
	\item Probleme beim Finden der richtigen Ableitung
	\item Einsatz im Bereich der Vorlesung Compilerbau I
    \end{itemgroup}

    \begin{itemgroup}{Lösung}
	\item Linksfaktorisierung der Grammatik
	\end{itemgroup}
    
    \vfill{}
}

\myslide{Perspektiven - Ableitungen für Grammatikenn}
{
    \begin{itemgroup}{}
	\item Wörter auch für Grammatiken überprüfbar
	\item Einsatz im Bereich der Vorlesung Compilerbau I
    \end{itemgroup}

    \begin{itemgroup}{Lösung}
	\item Finden der richtigen Ableitung durch rekursiven Abstieg
	\item Auflistung der verwendeten Produktionen
	\end{itemgroup}
    
    \vfill{}
}


\myslide{Perspektiven - Auto-Layout}
{
    \begin{itemgroup}{}
	\item Der Auto-Layout Algorithmus hat als Kostenfunktion nur Über\-schnei\-dung
	der Kanten
	\item Oft schlechte Ergebnisse, die durch den Nutzer korrigiert werden müssen
    \end{itemgroup}

    \begin{itemgroup}{Lösung}
	\item Algorithmus mit besserer Kostenfunktion (Simulated-Annealing)
	\item Kostenfunktion sollte mehrere Parameter berücksichtigen
		\begin{itemgroup}{}
		\item Überschneidung von Kanten
		\item Länge von Kanten
		\item Überscheidung von Kante und Zustand
		\end{itemgroup}
	\end{itemgroup}
    
    \vfill{}
}


%### removes texlipse warnings
