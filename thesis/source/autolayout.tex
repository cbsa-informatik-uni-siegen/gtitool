%%
%% $Id$
%%
%% Copyright (c) 2007-2008 Christian Fehler
%% Copyright (c) 2007-2008 Benjamin Mies
%%


\chapter{Auto Layout}\label{AutoLayout}

Beim Anlegen eines neuen Automaten, oder auch modifizieren eines bestehenden,
kommt es vor, dass die Übersichtlichkeit verloren geht. Damit man eine schnelle
Möglichkeit hat, den Automaten wieder besser überblicken zu können wurde eine
Autolayout Funktion implementiert.\vspace{10pt}

Diese Funktion basiert zur Zeit auf dem Erweiterten Partitionierungsalgorithmus
von Kerninghan und Lin , welcher auch als "`Min Cut"'- Algorithmus
bekannt ist. Er beruht auf iterativer Verbesserung durch paarweisen
Austausch.\vspace{10pt}

Dieser Algorithmus wird eigentlich für die Partitionierung von Schaltungen
verwendet, kann aber problemlos auf unsere Situation übertragen werden.
Allerdings mussten bei der Übertragung einige Modifikationen
vorgenmommen werden, auf welche ich später noch näher eingehen
möchte.\vspace{10pt}

\section{Der Kerninghan-Lin-Algorithmus im Details}

Zu Beginn wird das Partitionierungsproblem in eine Graphendarstellung
transformiert. Danach werden die Knoten des Graphen in zwei Gruppen unterteilt.
Anschließend werden für die aktuelle Einteilung die Schnittkosten berechnet.
Zur Berechnung der Schnittkosten werden die von der Schnittlinie, die Linie die
die beiden Gruppen trennt, erfassten Kanten gezählt.\vspace{10pt}

%### removes texlipse warning
Der Kerninghan-Lin-Algorithmus lässt sich in die folgenden Schritte
unterteilen, welche dem Buch "`Layoutsynthese elektronischer
Schaltungen"' (\cite{Layout}) entnommen wurden:\vspace{10pt}
%### removes texlipse warning

Schritt 0:
\begin{itemize}
  \item V = Menge der 2n Knoten 
  \item {A,B} sei eine willkürliche Anfangspartitionierung
\end{itemize}

Schritt 1:
\begin{itemize}
  \item i=1
  \item Berechnung von D(v) für alle Knoten v $\in$ V
\end{itemize}

Schritt 2:
\begin{itemize}
  \item Auswahl von $a_i$ und $b_i$ mit maximalem Gewinnwert $\Delta g_i =
  D(a_i) + D(b_i) - 2 * c(a_i b_i)$
  \item Vertauschen und fixieren von $a_i$ und $b_i$
\end{itemize}


Schritt 3:
\begin{itemize}
  \item Wenn alle Knoten fixiert sind, weiter mit Schritt 4, andernfalls
  \item Neuberechnung der D-Werte für alle Knoten, welche nicht fixiert und
  mit $a_i$ und $b_i$ verunden sind
  \item i=i+1
  \item Weiter mit Schritt 2
\end{itemize}

Schritt 4:
\begin{itemize}
  \item Bestimmung der Vertauschungssequenz 1 bis m $(1 \leq m \leq i)$, so dass
  $G_m = \sum_{i=1}^{m}{\Delta g_i}$ maximiert wird
  \item Wenn $G_m > 0$, weiter mit Schritt 5, andernfalls ENDE
\end{itemize}

Schritt 5:
\begin{itemize}
  \item Durchführen alle m Vertauschungen, Beseitigen aller Knotenfixierungen
  \item Weiter mit Schritt 1
\end{itemize}\vspace{10pt}

Schauen wir uns zunächst einmal an, wie der Algorithmus vorgeht, indem wir
nachvollziehen, was in den einzelnen Schritten passiert.\vspace{10pt}

In $Schritt\ 0$ werden die Knoten in zwei gleichgroße Gruppen unterteilt, wobei
es keine Rolle spielt, welcher Knoten in welcher Gruppe landet. Danach werden in
$Schritt\ 1$ für alle Knoten berechnet, ob sie einen hohen oder niedrigen Anteil
der Schnittkosten verursachen. Der Gewinnwert $\Delta g$, welcher in $Schritt\
2$ erwähnt wird, beschreibt die Verbesserung der Schnittkosten, welche durch
einen Knotentausch erreicht werden kann. Die beiden Knoten, welche den
maximalen Gewinnwert bringen, werden dann getauscht und fixiert. In $Schritt\ 3$
überprüfen wir ob alle Knoten bereits fixiert sind, ansonsten werden die Kosten
für jeden Knoten der noch nicht fixiert ist neu berechnet und erneut $Schritt\
2$ ausgeführt. In $Schritt\ 4$ wird überprüft ob der Algorithmus bereits sein
Ende erreicht hat. Wenn nicht, werden in $Schritt\ 5$ die Vertauschungen des
aktuellen Durchgangs durchgeführt, und es wird ein neuer Durchgang gestartet.\vspace{10pt}

%### removes texlipse warning
Wie schon erwähnt musste der Algorithmus für die Implementierung ein wenig
modifiziert werden. Zum einen  geht man eigentlich davon aus, dass die Knoten
in zwei etwa gleichgroße Gruppen einteilt. Dies erwies sich bei der
Implementierung allerdings als eher unvorteilhaft. Daher wird dynamisch
bestimmt, in wieviele Gruppen die Knoten unterteilt werden, wobei die Größe
aller Gruppen aber auch hier in etwa die gleiche ist. Diese
Implementierung entspricht soweit der Erweiterung des
Kerningham-Lin-Algorithmus wie sie in \cite{Layout} beschrieben
wird\vspace{10pt}
%### removes texlipse warning

Die Erweiterung des Algorithmus sieht jetzt vor, dass man den Algorithmus jetzt
für alle möglichen zweier Paarungen von Gruppen betrachtet. Die Implementierung
sieht so aus, dass man eine Linie zwischen die erste und zweite Gruppe legt,
und jetzt alle Kanten zählt, die diese Linie schneiden. Jetzt wird berechnet, bei welchen
beiden Knoten ein Tausch die größte Verbesserung liefert, also die meißten
Schnitte mit der Linie wegfallen. Diese werden dann getauscht und fixiert. Wenn
alle Knoten fixiert sind, oder keine Tauschkombination existiert bei der
Schnitte mit der Linie zwischen den Gruppen wegfallen, wird die Linie eine
Gruppe nach unten geschoben, also zwischen die zweite und dritte Gruppe und die
Vorgehensweise wiederholt. Der Algorithmus terminiert, nachdem die vorletzte und
letzte Gruppe abgearbeitet wurden.\vspace{10pt}

Dieser Algorithmus liefert noch nicht das optimale Ergebnis, was zum Beispiel
auch daran liegt, dass die Länge der Kanten in keinster Weise berücksichtigt
werden. Daher wird im Kapitel \ref{PerspectiveAutoLayout} ein alternativer
Algorithmus beschrieben, welcher wahrscheinlich bessere Ergebnisse für ein
automatischen Layout liefern könnte.
