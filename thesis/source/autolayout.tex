%%
%% $Id$
%%
%% Copyright (c) 2007-2008 Christian Fehler
%% Copyright (c) 2007-2008 Benjamin Mies
%%


\chapter{Auto Layout}\label{AutoLayout}

Beim Anlegen eines neuen Automaten, oder auch modifizieren eines bestehenden,
kommt es vor, dass die Übersichtlichkeit verloren geht. Damit man eine schnelle
Möglichkeit hat, den Automaten wieder besser überblicken zu können wurde eine
Autolayout Funktion implementiert.\vspace{10pt}

Diese Funktion basiert zur Zeit auf dem Erweiterten Partitionierungsalgorithmus
von Kerninghan und Lin, welcher auch als "`Min Cut"'- Algorithmus bekannt
ist. Er beruht auf iterativer Verbesserung durch paarweisen
Austausch.\vspace{10pt}

Dieser Algorithmus wird eigentlich für die Partitionierung von Schaltungen
verwendet, kann aber problemlos auf unsere Situation übertragen werden.
Allerdings mussten bei der Übertragung einige Modifikationen
vorgenmommen werden, auf welche ich später noch näher eingehen
möchte.\vspace{10pt}

\section{Der Kerninghan-Lin-Algorithmus im Details}

Zu Beginn wird das Partitionierungsproblem in eine Graphendarstellung
transformiert. Danach werden die Knoten des Graphen in zwei Gruppen unterteilt.
Anschließend werden für die aktuelle Einteilung die Schnittkosten berechnet.
Zur Berechnung der Schnittkosten werden die von der Schnittlinie, die Linie die
die beiden Gruppen trennt, erfassten Kanten gezählt.\vspace{10pt}

TODO BM explain algorithm


