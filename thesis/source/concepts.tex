%%
%% $Id$
%%
%% Copyright (c) 2007-2008 Christian Fehler
%% Copyright (c) 2007-2008 Benjamin Mies
%%


\chapter{Konzepte}\label{Concepts}

In diesem Kapitel werden alle wichtigen Konzepte besprochen, die bei der
Umsetzung des \gtitools berücksichtigt, bzw. erst während der Entwicklung
diskutiert und anschließend in die Tat umgesetzt wurden.\vspace{10pt}


\section{Unterstützung von Lerngruppen}\label{ConceptsLerning}

Ein wichtiges Konzept bei der Umsetzung war die Unterstützung von Lerngruppen.
Das \gtitool sollte somit in irgendeiner Weise dazu in der Lage sein, nicht nur
einem Benutzer zur Verfügung zu stehen, sondern sollte auch von einer
Lerngruppe benutzt werden können. Bei der Planung wurden verschiedene
Möglichkeiten diskutiert, auf welche Weise Lerngruppen unterstützt werden
können. Schließlich wurde beschlossen, den Benutzern einen Austausch von
geöffneten Automaten und Grammatiken zu ermöglichen.\vspace{10pt}

\begin{figure}[h!]
\begin{center}
\includegraphics[width=12cm]{../images/exchange.png}
\caption{Unterstützung von Lerngruppen}
\end{center}
\end{figure}
\vspace{10pt}

Der Austausch von Dateien wurde so umgesetzt, dass ein Benutzer seine
geöffneten Dateien an andere Benutzer verschicken kann, falls der Empfänger
zuvor das Empfangen eingeleitet hat. Somit sind Mitglieder von Lerngruppen in
der Lage Erkenntnisse mit den anderen auszutauschen. Dies sollte dazu dienen
das Verständnis der Materie zu verbessern.\vspace{10pt}


\section{Möglichst wenig Eingabebeschränkungen}\label{ConceptsInput}

Ein weiteres sehr wichtiges, wenn nicht sogar das wichtigste Konzept bei der
Umsetzung war die auch in \ref{InteractionErrorWarning} angesprochene
Vorgehensweise mit Benutzereingaben. Der Benutzer sollte so wenig wie möglich
in der Eingabe beschränkt werden. Ein Beispiel dafür ist, dass er in einem DEA
einen $\epsilon$-Übergang anlegen kann, obwohl dies für den Automaten nicht
erlaubt ist. Allerdings kann er einen solchen Automaten nicht benutzen, denn
bevor etwas mit dem Automaten, zum Beispiel eine Wort Eingabe, erfolgen kann,
wird der Automat automatisch validiert, wobei die in
\ref{InteractionErrorWarning} angegebenen Fehler und Warnungen auftreten
können.\vspace{10pt}

Hintergrund von diesem Konzept war, dass der Benutzer durch dieses Vorgehen mehr
lernt, als wenn er, um in dem Beispiel zu bleiben, einen $\epsilon$-Übergang gar
nicht erst hätte anlegen dürfen. Dann hätte er sich vermutlich nur gewundert,
warum kein \Symbol{$\epsilon$} der Übergangs-Menge hinzugefügt werden
kann.\vspace{10pt}
