%%
%% $Id$
%%
%% Copyright (c) 2007-2008 Christian Fehler
%% Copyright (c) 2007-2008 Benjamin Mies
%%


%### removes texlipse warnings


\chapter{Fazit}\label{Conclusion}

Das von uns implementierte \gtitool wurde entwickelt, um den Benutzern die
Grundlagen der theoretischen Informatik näher zu bringen. Ob dies gelungen ist,
wird der Einsatz der Software in der Zukunft zeigen. Um dieses Ziel zu erreichen,
wurden alle einzelnen Komponenten des Programms zusammen geplant und anschließend
eng verzahnt von uns beiden umgesetzt. Bei dieser Umsetzung war immer einer
hauptverantwortlich für den jeweiligen Bereich. Je nach Erfahrung ergänzten wir
uns allerdings bei der Umsetzung, um möglichst effizient arbeiten zu
können.\vspace{10pt}

Bei der Implementierung war es uns wichtig in etwa den gleichen Umfang des
Programms Machines (siehe \cite{machines}) zu erreichen oder so vorzubereiten,
dass er in Zukunft erreicht werden kann. Eine eigene Implementierung war
wichtig, da Machines nicht ganz zu der Vorlesung passte und in einigen Punkten
aus unserer Sicht verbesserungswürdig erschien. So werden Algorithmen zu wenig
detailreich umgesetzt und die Bedienung ist zum Teil wenig intuitiv. Aus diesen
Gründen war eine eigenständige Implementierung, die an die aktuelle Vorlesung
angepasst werden kann, sehr hilfreich.\vspace{10pt}

Bei der ersten Planung bzw. Vorbesprechung der Diplomarbeit wurden die Ziele in
den Vordergrund gestellt, verschiedene Automaten und Grammatiken zu
unterstützen. Während der Implemtierung wurden immer genauere Ziele gesteckt
bzw. bestehende Ziele eingegrenzt. So entstand die Umwandlung der Automaten in
andere Automaten oder die Umwandlung von Grammatiken in Automaten. Ein weiteres
Beispiel für die nachträgliche Anpassung der Ziele ist die Umsetzung der
erreichbaren Zustände. Die Notwendigkeit eines solchen Algorithmus entstand
durch das Umwandeln unter Verwendung eines Potenzautomaten. Da bei einer
solchen Umwandlung sehr große Automaten entstehen können, wurde das im
Abschnitt \ref{ReachableStates} vorgestellte Verfahren notwendig.\vspace{10pt}

Bei der Umsetzung des \gtitools war es uns sehr wichtig, dass wir eine
qualitativ hochwertige Software entwickeln. Dies betrifft sowohl das
Aussehen des Programms, aber auch den Source Code. Durch diese angestrebte hohe
Qualität wurde in Kauf genommen, dass quantitativ nicht alle Komponenten
umgesetzt werden konnten.\vspace{10pt}

Im Kapitel \ref{Perspective} wurden einige Perspektiven für die Zukunft
vorgestellt, da wir davon ausgehen, dass unsere Software in Zukunft
weiterentwickelt wird, um dem Benutzer eine noch bessere Unterstützung zur
Verfügung zu stellen. Der Nachteil quantitativ weniger erreicht zu haben wurde
damit gerechtfertigt, dass die Software leichter zu erweitern ist, wodurch
zukünftige Erweiterungen schneller und weniger fehleranfällig integriert werden
können. Ein weiterer Vorteil dieser Umsetzung ist, dass in Zukunft auftauchende
Fehler leichter und somit schneller beseitigt werden können.\vspace{10pt}

Abschließend bleibt zu sagen, dass die Planung und die anschließende
Implementierung des \gtitools sehr viel Freude gemacht hat. Wir hoffen, dass sich
der Aufwand gelohnt hat und, dass wir den Studenten die theoretischen Inhalte der
Informatik näher gebracht haben. Gleichfalls würden wir uns freuen, wenn unsere
Software in Zukunft weiterentwickelt würde, um weitere Funktionen zu
ergänzen.\vspace{10pt}


%### removes texlipse warnings
