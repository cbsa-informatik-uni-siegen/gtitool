%%
%% $Id$
%%
%% Copyright (c) 2007-2008 Christian Fehler
%% Copyright (c) 2007-2008 Benjamin Mies
%%


\chapter{Umwandeln}\label{ConverTo}

In diesem Kapitel soll das Umwandeln von einer Grammatik in einen Automaten
bzw. zwischen den verschiedenen Automaten Typen besprochen werden. Dabei wird
im Besonderen darauf eingegangen, welche Herausforderungen aufgetreten sind, um
dem Benutzer die verwendeten Algorithmen möglichst in kleinen Schritten und
verständlich darzustellen. Beim Umwandeln der Grammatiken in die entsprechenden
Automaten sind die verwendeten Algorithmen nicht besonders gut darzustellen,
weshalb hier auf eine schrittweise Umwandlung verzichtet wurde.


\section{Grammatik umwandeln}

TODOBM


\section{Automat umwandeln}

Bei der Planung der Umwandlung zwischen den verschiedenen Automaten Typen
stellte sich die Frage, wie dem Benutzer der Umwandlungsalgorithmus möglichst
verständlich dargestellt wird. Ich entschied mich für das auch sonst im
\gtitool verwendete Verfahren, eine Navigationsleiste zu verwendet, die es dem
Benutzer gestattet in dem Algorithmus einen Schritt vor, bzw. einen Schritt
zurück zu gehen. Um die Handhabung für den Benutzer zu erleichtern, wurde die
Navigationsleiste so erweitert, dass der Benutzer zurück an den Anfang der
Ausführung des Algorithmus springen kann, ebenfalls an das Ende und, dass die
einzelnen Schritte auch automatisch, nach einer einstellbaren Zeit, durchgeführt
werden können.\vspace{10pt}

Das ist ein Test (vgl. \cite{Sieber0}).