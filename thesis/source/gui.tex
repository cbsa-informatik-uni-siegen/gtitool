%%
%% $Id$
%%
%% Copyright (c) 2007-2008 Christian Fehler
%% Copyright (c) 2007-2008 Benjamin Mies
%%


\chapter{Oberflächengestaltung}\label{GUI}


\section{Redo/Undo}

Da das Lernwerkzeug vom Aussehen her sehr an einen Editor angelehnt ist,
wollten wir dem Benutzer auch eine Möglichkeit geben Schritte rückgängig zu
machen oder wiederherzustellen. Dies war jedoch zu einem Zeitpunkt, zu dem bei
weitem noch nicht alle Funktionalität implementiert war. Desweiteren soll das
Werkzeug ja auch zukünftig an den Stoff der Vorlesung angepasst werden, so dass
uns eine einfache Erweiterbarkeit sehr wichtig war.\vspace{10pt}

Daher haben wir uns entschlossen die Verwaltung der Redo/Undo Schritte nicht
einer Klasse alleine zu überlassen, sondern die verschiedenen Aktionen in
einzelnen Items zu kapseln. Auf diesen Items kann dann die entsprechende redo
bzw. undo Funktion aufgerufen werden. Verwaltet wird dies von einem Handler,
welcher sich nur merkt welches das letzte aktive Item für Redo und Undo ist,
und den Aufruf durch den Nutzer an das Item weiterleitet.\vspace{10pt}

Durch diese Kapselung ist es leicht möglich die Redo/Undo Funktion zu
erweitern, da einfach ein neues Item für die entsprechende Aktion implementiert
werden muss, um diese Item beim Ausführen der Aktion an den Handler zu
übergeben.


\section{Anpassung aller GUI Komponenten}

TODOCF


\section{Zweite Ansicht}

TODOCF