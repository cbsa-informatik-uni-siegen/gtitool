%%
%% $Id$
%%
%% Copyright (c) 2007-2008 Christian Fehler
%% Copyright (c) 2007-2008 Benjamin Mies
%%


\chapter{Interaktion mit dem Benutzer}\label{Interaction}


\section{Fehler und Warnung}


\subsection{Automat}

TODOCF


\subsection{Grammatik}

Im folgenden möchten wir auf die Fehler und Warnungen eingehen, welche bei dem
Validierungsvorgang einer Grammatik auftreten können.\vspace{10pt}

Als Validierungsfehler zählt, unter anderem, wenn die gleiche Produktion mehrfach
angelegt wurde. In der Spalte Meldung sieht man wie immer nur, um welchen Fehler
es sich gerade handelt. Über die Beschreibung wird mitgeteilt, um welche
Produktion es sich dabei handelt. Man kann sich die betreffenden Produktionen
auch farblich hervorheben lassen, indem man die entsprechende Fehlermeldung durch
Mausklick auswählt. So kann der Benutzer die fehlerhaften Produktionen schnell
lokalisieren und den Fehler beheben.\vspace{10pt}

Wenn es sich bei der aktuellen Grammatik um eine reguläre Grammatik handelt
gibt es noch eine weitere Fehlermeldung. Wenn nicht alle Produktion dem
vorgegebenen Muster für eine rechtsreguläre Grammtik entsprechen, wird dies auch
über einen Validierungsfehler angezeigt. Dabei kann man auch hier anhand der
Beschreibung sehen, um welche Produktion es sich handelt. Wenn man den
entsprechenden Eintrag auswählt, wird dem Benutzer wieder durch farbiges
hervorheben signalisiert wo sich diese Produktion befindet. Es wird jetzt
allerdings nicht die komplette Produktion eingefärbt, sondern nur der Teil der
Satzform, welcher nicht dem Muster entspricht. Auch hier war wieder die
Intention dem Benutzer schnell zu einer gültigen Grammatik zu
verhelfen.\vspace{10pt}

Neben diesen beiden Fehlern haben wir auch eine Warnung eingeführt. Diese
gibt Auskunft darüber, wenn ein Nonterminal, vom Startsymbol ausgehend,
nicht erreichbar ist. Es handelt sich hierbei nur um eine Warnung, weil es
sich durchaus um eine gültige Grammatik handeln kann. Sie soll dem Benutzer
allerdings dazu animieren, die Übergangsmenge zu kontrollieren, ob nicht eine,
oder mehrere, Produktionen vergessen wurden. So soll es möglich sein
Flüchtigkeitsfehler frühzeitig zu erkennen und zu beheben.


\section{Operationen mit dem Automaten Keller}

TODOCF