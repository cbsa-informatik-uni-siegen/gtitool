%%
%% $Id$
%%
%% Copyright (c) 2007-2008 Christian Fehler
%% Copyright (c) 2007-2008 Benjamin Mies
%%


\chapter{Einleitung}\label{Introduction}

Das Fach Grundlagen der theoretischen Informatik ist für alle
Informatikstudenten der Universität Siegen eine Pflichtveranstaltung. Allerdings
haben bisherige Klausuregebnisse gezeigt, dass einige Studenten Probleme mit diesem
Themengebiet haben.\vspace{10pt}

Um die Studenten bei diesen Problemen zu untersützen, ist die Idee entstanden
ein Lernwerkzeug für dieses Fach zu entwickeln. Dieses Lernwerkzeug soll den
Stoff der Vorlesung und Übungen abdecken, und den Studenten helfen diesen
besser nachvollziehen zu können, und Verständnisprobleme zu
beseitigen.\vspace{10pt}

Dieses Lernwerkzeug haben wir im Umfang unserer Diplomarbeit entwickelt, und
wollen im Folgenden vorstellen, welche Funktionen bereits verfügbar sind.
Weiterhin möchten wir in dem Kapitel \ref{Perspective} Vorschläge für Erweiterungen
dieses Lernwerkzeugs nennen, wobei wir auch schon grobe Umsetzungsvorschläge
mit angeben werden.\vspace{10pt}
