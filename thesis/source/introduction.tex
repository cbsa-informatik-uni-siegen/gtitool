%%
%% $Id$
%%
%% Copyright (c) 2007-2008 Christian Fehler
%% Copyright (c) 2007-2008 Benjamin Mies
%%


%### removes texlipse warnings


\chapter{Einleitung}\label{Introduction}

Das Fach Grundlagen der theoretischen Informatik (\textit{GTI}) ist für alle
Informatikstudenten der Universität Siegen eine Pflichtveranstaltung. Allerdings
haben bisherige Klausurergebnisse gezeigt, dass einige Studenten Probleme mit
diesem Themengebiet haben.\vspace{10pt}

Um die Studenten bei der Lösung dieser Probleme zu unterstützen, ist die Idee
entstanden, ein Lernwerkzeug (im weiteren Text als \textit{\gtitool} bezeichnet)
für dieses Fach zu entwickeln. Dieses Lernwerkzeug soll den Stoff der Vorlesung
und der Übungen abdecken, und den Studenten helfen, diesen besser
nachvollziehen zu können, um Verständnisprobleme zu beseitigen.\vspace{10pt}

\noindent
Die Vorlesung GTI kann in die folgenden Themengebietet unterteilt
werden:\vspace{10pt} 

\begin{itemize}
  \item Reguläre Sprachen und endliche Automaten
  \item Kontextfreie Sprachen und Kellerautomaten
  \item Berechenbarkeitstheorie
\end{itemize}\vspace{10pt}

\noindent
Der Umfang des \gtitools beschränkt sich auf Auszüge der Kapitel
"`Reguläre Sprachen und endliche Automaten"' und "`Kontextfreie Sprachen und
Kellerautomaten"'.\vspace{10pt}

Der Benutzer hat die Möglichkeit eigene Automaten, Kapitel \ref{Machines}, und
Grammatiken, Kapitel \ref{Grammars},  zu definieren. Die definierten Grammatiken
können in Automaten konvertiert werden, was in Abschnitt \ref{ConverToGrammar}
beschrieben wird. Für die Automaten besteht die Möglichkeit, diese in einen
anderen Automatentypen umzuwandeln, worauf wir in Abschnitt \ref{ConverToMachine}
eingehen werden. In Abschnitt \ref{WordNavigationDeterministic} werden wir
beschreiben, wie ein Benutzer überprüfen kann, ob ein Wort in der von einem
Automaten erkannten Sprache liegt. Der Benutzer hat auch die Möglichkeit, sich
aus einem definierten Automaten einen minimalen Automaten konstruieren zu lassen,
welcher die gleiche Sprache erkennt. Diese Minimierung wird in Abschnitt
\ref{Minimize} erklärt.\vspace{10pt}

Während der Diplomarbeit haben wir einige Konzepte diskutiert und festgelegt,
welche bei der Implementierung in die Tat umgesetzt wurden. Ein wichtiges Konzept
bei der Umsetzung war die Unterstützung von Lerngruppen. Das \gtitool sollte
somit in irgendeiner Weise dazu in der Lage sein, nicht nur einem einzelnen
Benutzer zur Verfügung zu stehen, sondern sollte auch von einer Lerngruppe
benutzt werden können. Bei der Planung wurden verschiedene Möglichkeiten
diskutiert, auf welche Weise wir Lerngruppen unterstützen können. Schließlich
wurde beschlossen, den Benutzern einen Austausch von Automaten und Grammatiken zu
ermöglichen.\vspace{10pt}

Der Austausch von Dateien wurde so umgesetzt, dass ein Benutzer seine geöffneten
Dateien an andere Benutzer verschicken kann, falls der Empfänger zuvor das
Empfangen eingeleitet hat. Somit sind Mitglieder von Lerngruppen in der Lage,
Erkenntnisse mit den Anderen auszutauschen. Dies sollte dazu dienen, das
Verständnis der Materie zu verbessern.\vspace{10pt}

Ein weiteres sehr wichtiges, wenn nicht sogar das wichtigste Konzept bei der
Umsetzung, war, die Vorgehensweise mit Benutzereingaben. Der Benutzer sollte so
wenig wie möglich in der Eingabe beschränkt werden. Ein Beispiel dafür ist, dass
er in einem DEA einen $\epsilon$-Übergang anlegen kann, obwohl das für dieses
Automatenmodell nicht erlaubt ist. Allerdings kann er einen solchen Automaten
erst benutzen, wenn dieser auf Korrektheit überprüft wurde, und alle
festgestellten Fehler beseitigt wurden. Diese Überprüfung auf Korrektheit wird
im Folgenden als \textit{Validierung} bezeichnet, welche im \gtitool für
Automaten und Grammatiken möglich ist.\vspace{10pt}

Hintergrund von diesem Konzept war, dass der Benutzer durch dieses Vorgehen mehr
lernt, als würde er, um in dem Beispiel zu bleiben, einen $\epsilon$-Übergang
gar nicht erst anlegen dürfen. Dann hätte er sich vermutlich nur gewundert,
warum er \Symbol{$\epsilon$} nicht der Übergangs-Menge hinzufügen
kann.\vspace{10pt}


%### removes texlipse warnings
