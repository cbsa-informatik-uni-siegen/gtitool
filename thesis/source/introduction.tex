%%
%% $Id$
%%
%% Copyright (c) 2007-2008 Christian Fehler
%% Copyright (c) 2007-2008 Benjamin Mies
%%


%### removes texlipse warnings


\chapter{Einleitung}\label{Introduction}

Das Fach Grundlagen der theoretischen Informatik ist für alle
Informatikstudenten der Universität Siegen eine Pflichtveranstaltung. Allerdings
haben bisherige Klausurergebnisse gezeigt, dass einige Studenten Probleme mit
diesem Themengebiet haben.\vspace{10pt}

Um die Studenten bei der Lösung dieser Probleme zu unterstützen, ist die Idee
entstanden, ein Lernwerkzeug für dieses Fach zu entwickeln. Dieses Lernwerkzeug
soll den Stoff der Vorlesung und der Übungen abdecken, und den Studenten helfen,
diesen besser nachvollziehen zu können, um Verständnisprobleme zu
beseitigen.\vspace{10pt}

Dieses Lernwerkzeug haben wir im Rahmen unserer Diplomarbeit entwickelt, und im
Folgenden soll vorgestellt werden, welche Funktionen bereits verfügbar sind. In
Kapitel \ref{Perspective} werden wir noch auf zukünftige Aussichten für dieses
Projekt eingehen, wobei wir Erweiterungsmöglichkeiten nennen und einen Vorschlag
für die Realisierung geben werden.\vspace{10pt}


%### removes texlipse warnings
