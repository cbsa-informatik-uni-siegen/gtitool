%%
%% $Id$
%%
%% Copyright (c) 2007-2008 Christian Fehler
%% Copyright (c) 2007-2008 Benjamin Mies
%%


%### removes texlipse warnings


\chapter{Einleitung}\label{Introduction}

Das Fach Grundlagen der theoretischen Informatik ist für alle
Informatikstudenten der Universität Siegen eine Pflichtveranstaltung. Allerdings
haben bisherige Klausurergebnisse gezeigt, dass einige Studenten Probleme mit
diesem Themengebiet haben.\vspace{10pt}

Um die Studenten bei der Lösung dieser Probleme zu unterstützen, ist die Idee
entstanden, ein Lernwerkzeug für dieses Fach zu entwickeln. Dieses Lernwerkzeug
soll den Stoff der Vorlesung und der Übungen abdecken, und den Studenten helfen,
diesen besser nachvollziehen zu können, um Verständnisprobleme zu
beseitigen.\vspace{10pt}

Dieses Lernwerkzeug haben wir im Rahmen unserer Diplomarbeit entwickelt, und im
Folgenden soll vorgestellt werden, welche Funktionen bereits verfügbar sind. In
Kapitel \ref{Perspective} werden wir noch auf zukünftige Aussichten für dieses
Projekt eingehen, wobei wir Erweiterungsmöglichkeiten nennen und einen Vorschlag
für die Realisierung geben werden.\vspace{10pt}

TODO

In diesem Kapitel werden alle wichtigen Konzepte besprochen, die bei der
Umsetzung des \gtitools berücksichtigt bzw. erst während der Entwicklung
diskutiert und anschließend in die Tat umgesetzt wurden.\vspace{10pt}

Ein wichtiges Konzept bei der Umsetzung war die Unterstützung von Lerngruppen.
Das \gtitool sollte somit in irgendeiner Weise dazu in der Lage sein, nicht nur
einem einzelnen Benutzer zur Verfügung zu stehen, sondern sollte auch von einer
Lerngruppe benutzt werden können. Bei der Planung wurden verschiedene
Möglichkeiten diskutiert, auf welche Weise wir Lerngruppen unterstützen
können. Schließlich wurde beschlossen, den Benutzern einen Austausch von
Automaten und Grammatiken zu ermöglichen.\vspace{10pt}

Der Austausch von Dateien wurde so umgesetzt, dass ein Benutzer seine geöffneten
Dateien an andere Benutzer verschicken kann, falls der Empfänger zuvor das
Empfangen eingeleitet hat. Somit sind Mitglieder von Lerngruppen in der Lage,
Erkenntnisse mit den Anderen auszutauschen. Dies sollte dazu dienen, das
Verständnis der Materie zu verbessern.\vspace{10pt}

Ein weiteres sehr wichtiges, wenn nicht sogar das wichtigste Konzept bei der
Umsetzung, war, die bereits in Abschnitt \ref{InteractionErrorWarning}
angesprochene Vorgehensweise mit Benutzereingaben. Der Benutzer sollte so wenig
wie möglich in der Eingabe beschränkt werden. Ein Beispiel dafür ist, dass er in
einem DEA einen $\epsilon$-Übergang anlegen kann, obwohl das für dieses
Automatenmodell nicht erlaubt ist. Allerdings kann er einen solchen Automaten
nicht benutzen, denn bevor etwas ihm, zum Beispiel eine Wort-Navigation, erfolgen
kann, wird der Automat automatisch validiert, wobei die in Abschnitt
\ref{InteractionErrorWarning} angegebenen Fehler und Warnungen auftreten
können.\vspace{10pt}

Hintergrund von diesem Konzept war, dass der Benutzer durch dieses Vorgehen mehr
lernt, als würde er, um in dem Beispiel zu bleiben, einen $\epsilon$-Übergang
gar nicht erst anlegen dürfen. Dann hätte er sich vermutlich nur gewundert,
warum er \Symbol{$\epsilon$} nicht der Übergangs-Menge hinzufügen
kann.\vspace{10pt}


%### removes texlipse warnings
