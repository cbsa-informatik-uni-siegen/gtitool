%%
%% $Id$
%%
%% Copyright (c) 2007-2008 Christian Fehler
%% Copyright (c) 2007-2008 Benjamin Mies
%%


%### removes texlipse warnings


\chapter{Einleitung}\label{Introduction}

Das Fach {\em Grundlagen der theoretischen Informatik} (kurz: GTI) ist eine
Pflichtveranstaltung für alle Informatikstudenten der Universität Siegen.
Allerdings haben bisherige Klausurergebnisse gezeigt, dass einige Studenten
Probleme mit diesem Themengebiet haben.\vspace{10pt}

Um die Studenten bei der Lösung dieser Probleme zu unterstützen, ist die Idee
entstanden, ein Lernwerkzeug (im weiteren Text als {\em \gtitool} bezeichnet)
für dieses Fach zu entwickeln. Dieses Lernwerkzeug soll den Stoff der Vorlesung
und der Übungen abdecken, und den Studenten helfen, diesen besser
nachvollziehen zu können, um Verständnisprobleme zu beseitigen.\vspace{10pt}

\noindent
Die Vorlesung GTI ist unterteilt in folgende Themengebiete:\vspace{10pt} 

\begin{itemize}
  \item Reguläre Sprachen und endliche Automaten
  \item Kontextfreie Sprachen und Kellerautomaten
  \item Berechenbarkeitstheorie
\end{itemize}\vspace{10pt}

\noindent
Der Umfang des \gtitools beschränkt sich auf Auszüge der Kapitel
"`Reguläre Sprachen und endliche Automaten"' und "`Kontextfreie Sprachen und
Kellerautomaten"'.\vspace{10pt}

Der Benutzer hat die Möglichkeit eigene Automaten, Kapitel \ref{Machines}, und
Grammatiken, Kapitel \ref{Grammars},  zu definieren. Die definierten Grammatiken
können in Automaten konvertiert werden, was in Kapitel \ref{ConverToGrammar}
beschrieben wird. Diese wiederum können in einen äquivalenten Automaten
eines anderen Typs konvertiert werden, worauf wir in Abschnitt 
\ref{ConverToMachine} eingehen werden. \vspace{10pt}


Die weiteren Abschnitte beschreiben einige spezielle Operationen, die auf den
Automaten durchgeführt werden können. Beispielsweise kann ein Benutzer interaktiv
überprüfen, ob ein Wort von einem Automaten akzeptiert wird (siehe Abschnitt
\ref{WordNavigationDeterministic}). Weiterhin hat der Benutzer die Möglichkeit,
sich aus einem definierten Automaten einen minimalen Automaten konstruieren zu
lassen, welcher die gleiche Sprache erkennt. Diese Minimierung wird in Abschnitt
\ref{Minimize} erklärt.\vspace{10pt}

Wir setzen im Folgenden voraus, dass der Leser mit diesen Konzepten vertraut
ist. Eine Einleitung in die zugrundliegende Thematik findet sich in
\cite{Schoening}.\vspace{10pt}

Während der Diplomarbeit haben wir einige Konzepte diskutiert und festgelegt,
welche bei der Implementierung in die Tat umgesetzt werden sollten. Ein wichtiges Konzept
bei der Umsetzung war die Unterstützung von Lerngruppen. Das \gtitool sollte
somit in irgendeiner Weise dazu in der Lage sein, nicht nur einem einzelnen
Benutzer zur Verfügung zu stehen, sondern sollte auch von einer Lerngruppe
benutzt werden können. Bei der Planung wurden verschiedene Möglichkeiten
diskutiert, auf welche Weise wir Lerngruppen unterstützen können. Schließlich
wurde beschlossen, den Benutzern einen Austausch von Automaten und Grammatiken zu
ermöglichen.\vspace{10pt}

Der Austausch von Dateien wurde so umgesetzt, dass ein Benutzer seine geöffneten
Dateien an andere Benutzer verschicken kann, falls der Empfänger zuvor das
Empfangen eingeleitet hat. Somit sind Mitglieder von Lerngruppen in der Lage,
Erkenntnisse mit den anderen auszutauschen. Dies sollte dazu dienen, das
Verständnis der Materie zu verbessern.\vspace{10pt}

Ein weiteres sehr wichtiges, wenn nicht sogar das wichtigste Konzept bei der
Umsetzung, war, die Vorgehensweise mit Benutzereingaben. Der Benutzer sollte so
wenig wie möglich in der Eingabe beschränkt werden. Ein Beispiel dafür ist, dass
er in einem DEA einen $\epsilon$-Übergang anlegen kann, obwohl das für dieses
Automatenmodell nicht erlaubt ist. Allerdings kann er einen solchen Automaten
erst benutzen, wenn dieser auf Korrektheit überprüft wurde, und alle
festgestellten Fehler beseitigt wurden. Diese Überprüfung auf Korrektheit wird im
Rahmen des Lernwerkzeugs als {\em Validierung} bezeichnet, und wird für Automaten
und Grammatiken unterstützt.\vspace{10pt}

Der Benutzer soll hierdurch in die Lage versetzt werden, aus seinen eigenen
Fehlern zu lernen. Denn der Lerneffekt ist allgemein höher, wenn er zunächst
einen Fehler machen kann, der ihm dann aufgezeigt wird, als würde man die
Möglichkeit diesen Fehler zu machen von vorneherein ausschliessen.\vspace{10pt}

Aus zeitlichen Gründen war es nicht möglich den Stoff der Vorlesung
voll\-ständig umzusetzen. Kapitel \ref{Perspective} beschreibt die bislang nicht
implementierten Funktionen, und zeigt mögliche Realisierungen auf.


%### removes texlipse warnings
