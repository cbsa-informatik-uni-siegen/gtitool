%%
%% $Id$
%%
%% Copyright (c) 2007-2008 Christian Fehler
%% Copyright (c) 2007-2008 Benjamin Mies
%%


\chapter{Automaten}\label{Machines}


\section{Graphenansicht}


\subsection{Automatendarstellung mit JGraph}

TODOBM


\subsection{Anpassung von JGraph}

TODOCF Das ist ein Test (vgl. \cite{Sieber0}).


\section{Tabellen}


\subsection{Übergangstabelle}

TODOCF


\subsection{Keller Operationen Tabelle}

TODOBM


\section{Wort-Navigation}


\subsection{Deterministische Navigation}

TODOBM


\subsection{Navigation im Kellerautomaten mit Auswahl}

TODOCF


\subsection{Zustands Pfad}

Bei Verwendung eines nicht deterministischen Automaten stellte sich bei der
Wort Navigation die Frage, auf welchem Pfad die aktuell aktiven Zustände
erreicht wurden. Es wurden verschiedene Umsetzungen in Betracht gezogen, dem
Benutzer die Ausgabe darzustellen. Es wurde schließlich entschieden, die
Zustände auf dem Weg zu den aktuell aktiven Zustanden darzustellen. Zwischen
diesen Zustanden werden die verwendeten Übergänge, sowie die verwendeten
Symbole an den Übergängen dargestellt, bzw. hervorgehoben. Da unter Umständen
viele Zustands Pfade vorhanden sein können, werden diese anhand der Anzahl der
verwendeten Zustände sortiert, somit werden die kürzesten Pfade zuerst
angezeigt.


\section{Erreichbare Zustände}\label{ReachableStates}

TODOCF