%%
%% $Id$
%%
%% Copyright (c) 2007-2008 Christian Fehler
%% Copyright (c) 2007-2008 Benjamin Mies
%%


\chapter{Automaten}\label{Machines}


\section{Graphenansicht}


\subsection{Automatendarstellung mit JGraph}

TODOBM


\subsection{Anpassung von JGraph}

TODOCF Das ist ein Test (vgl. \cite{Sieber0}).


\section{Tabellen}


\subsection{Übergangstabelle}

TODOCF


\subsection{Keller Operationen Tabelle}

TODOBM


\section{Wort-Navigation}


\subsection{Deterministische Navigation}

TODOBM


\subsection{Navigation im Kellerautomaten mit Auswahl}

TODOCF


\subsection{Zustands Pfad}

Bei Verwendung eines nicht deterministischen Automaten stellte sich bei der
Wort Navigation die Frage, auf welchem Pfad die aktuell aktiven Zustände
erreicht wurden. Es wurden verschiedene Umsetzungen in Betracht gezogen, dem
Benutzer die Ausgabe darzustellen. Es wurde schließlich entschieden, die
Zustände auf dem Weg zu den aktuell aktiven Zuständen darzustellen. Zwischen
diesen Zuständen werden die verwendeten Übergänge, sowie die verwendeten
Symbole an den Übergängen dargestellt, bzw. hervorgehoben. Da unter Umständen
viele Zustands Pfade vorhanden sein können, werden diese anhand der Anzahl der
verwendeten Zustände sortiert, somit werden die kürzesten Pfade zuerst
angezeigt.\vspace{10pt}

Der implementierte Algorithmus geht rückwärts vor, startet also bei den aktuell
aktiven Zuständen und geht alle Pfade zurück, bis alle, bis jetzt gelesenen
Symbole abgearbeitet sind. Startet der berechnete Pfad dann in dem
Startzustand, wurde ein gültiger Pfad erkannt. Bei der Implementierung des
Algorithmus musste eine Zykluserkennung umgesetzt werden, da es sonst möglich
war in eine Endlosschleife zu geraten, welche durch $\epsilon$-Übergänge
verursacht wurde.


\section{Erreichbare Zustände}\label{ReachableStates}

Nach dem in \ref{ConverToMachine} nachzulesenen Umwandeln eines nicht
deterministischen in einen deterministischen Automaten unter Verwendung der
Potenzautomatenkonstruktion stellte sich die Frage, wie die nicht erreichbaren
Zustände erkannt und entfernt werden können. Dieses Problem betrifft aber nicht
nur umgewandelte Automaten, sondern ganz allgemein, jeden erstellten
Automaten.\vspace{10pt}

Bei der Umsetzung sollte im Vordergrund stehen, dass der Benutzer mit sehr
kleinen Schritten verdeutlicht bekommt, wie der zugrunde liegende Algorithmus
arbeitet, so dass diese Arbeitsweise sehr einfach nachzuvollziehen
ist.\vspace{10pt}

Der verwendete Algorithmus besteht aus drei Phasen. In der ersten Phase wird ein
noch nicht abgearbeiteter Zustand ausgewählt, zu Beginn wird mit dem Startzustand
begonnen. In der zweiten Phase werden die von diesem Zustand aus direkt
erreichbaren Zustände berechnet und dem Benutzer hervorgehoben dargestellt. In
der dritten Phase werden alle bis jetzt zu erreichenden Zustände hervorgehoben.
Gleichzeitig bekommt der Benutzer in der Outline mitgeteilt, welcher Zustand
jetzt fertiggestellt ist und welche noch berechnet werden müssen. Durch diese
sehr feine Unterteilung, soll ein besseres Verständnis des Algorithmus erreicht
werden.
