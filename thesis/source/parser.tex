%%
%% $Id$
%%
%% Copyright (c) 2007-2008 Christian Fehler
%% Copyright (c) 2007-2008 Benjamin Mies
%%


\chapter{Parser}\label{Parser}

TODOCF: Parser definieren

Zu Beginn der Planungen für das \gtitool stellte sich die Frage, wie der Benutzer
die unterschiedlichen Eingaben machen soll. Es wurden verschiedene
Möglichkeiten in Betracht gezogen, schließlich wurde aber die Verwendung von
Parsern bevorzugt. Der Vorteil von Parsern ist, dass der Benutzer in seinen
Eingaben, für zum Beispiel ein Alphabet, so wenig wie möglich beschränkt wird.
Dies wäre nicht der Fall gewesen, wenn nur eine Auswahl von Zeichen zur
Verfügung gestanden hätte.\vspace{10pt}

%### removes texlipse warning
Zur Erzeugung der Parser wurde der CUP Parser-Generator für Java verwendet,
siehe \cite{java-cup}. Als Lexer wurde wegen dem guten Zusammenspiel mit dem
verwendeten Parser JFlex benutzt, siehe \cite{jflex}.\vspace{10pt}
%### removes texlipse warning

In diesem Kapitel werden die Besonderheiten der Parser angesprochen, vor allem
das Prüfen der kontextsensitiven Bedingungen und das Anpassen der verwendeten
Darstellungsweise. Desweiteren müssen auch Bedingungen geprüft werden, die im
Zusammenhang mit dem verwendeten Automaten bzw. der Grammatik
stehen.\vspace{10pt}

Wird eine solche Bedingung verletzt, kann der Benutzer geöffnete Dialoge, wie zum
Beispiel den Dialog zum Ändern des Alphabets, nicht mehr bestätigen und ist
gezwungen, den Fehler zu beheben, oder die Bearbeitung abzubrechen. Es besteht
also kein Unterschied zwischen der Behandlung eines Fehlers im Scanner, Parser
oder den kontextsensitiven Bedingungen.\vspace{10pt}


\section{Kontextsensitive Bedingungen}\label{ParserContext}

In diesem Abschnitt geht es darum, welche kontextsensitiven Bedingungen
über\-prüft werden müssen, um dem Benutzer zu verdeutlichen, welche Eingaben im
aktuellen Kontext nicht vorgenommen werden können. Ein wichtiger Aspekt dabei
ist, dass der Benutzer darauf hingewiesen wird, warum seine Eingabe abgelehnt
wird.\vspace{10pt}

Im \gtitool gibt es an verschiedenen Stellen die Möglichkeit ein Alphabet
einzugeben. Betrachten wir als erstes das Eingeben des Alphabetes im Dialog für
die Einstellungen, da dieser Parser die wenigsten kontextsensitiven Bedingungen
berücksichtigen muss. Sowohl beim Eingabealphabet, wie auch beim Kelleralphabet
ist die einzige Bedingung, dass ein Symbol nicht doppelt vorkommen darf, eine
Eingabe \{\Symbol{0}, \Symbol{0}\} ist somit nicht zulässig.\vspace{10pt}

Ebenfalls in den Einstellungen zu finden ist die Eingabe der
Nichtterminalzeichen, Terminalzeichen und des Startsymbols. Genau wie beim
Alphabet, darf auch hier ein Nichtterminalzeichen bzw. ein Terminalzeichen nur
einmal in der Menge vorkommen. Zusätzlich muss aber gelten, dass die Menge der
Nichtterminalzeichen  und die Menge der Terminalzeichen disjunkt sein müssen. Es
ist dem Benutzer somit nicht erlaubt bei den Nichtterminalzeichen
\{\NonterminalSymbol{E}, \NonterminalSymbol{a}\} und gleichzeitig
\{\TerminalSymbol{a}, \TerminalSymbol{b}\} bei den Terminalzeichen einzutragen.
Versucht der Benutzer eine solche Eingabe vorzunehmen, bekommt er zum Beispiel
bei den Nichtterminalzeichen angezeigt, dass \NonterminalSymbol{a} bereits ein
Terminalzeichen ist, bzw. umgekehrt, wenn zuerst das Nichtterminalzeichen
vorhanden war. Eine weitere Bedingung ist, dass das Startsymbol in der Menge der
Nichtterminalzeichen enthalten sein muss, auch diese Bedingung wird überprüft
und bei Verletzung mit einer Fehlermeldung behandelt.\vspace{10pt}

Eine weitere Kontextbedingung wird überprüft, wenn das Alphabet eines
bestehenden Automaten bearbeitet wird. Dabei darf der Benutzer beliebige, aber
noch nicht verwendete Symbole ergänzen. Er darf allerdings keine Symbole
entfernen, die noch von einem Übergang verwendet werden. Soll dies geschehen,
muss der Benutzer erst das Symbol aus allen Übergängen entfernen,
anschließend kann es aus dem Alphabet entfernt werden.\vspace{10pt}

Gleiches gilt für das Bearbeiten der Nichtterminalzeichen und Terminalzeichen
einer bestehenden Grammatik. Auch hier wird überprüft, ob das Zeichen in einer
der Produktionen benutzt wird. Ist dies der Fall, kann es nicht entfernt
werden.\vspace{10pt}

Ein ähnliches Prinzip wird bei einigen der übrigen Parsern verwendet, so zum
Beispiel das Keller Lese-Wort bzw. das Keller Schreib-Wort. Auch dort müssen
die verwendeten Symbole im Keller Alphabet vorkommen.\vspace{10pt}


\section{Anpassung der Darstellungsweise}\label{ParserAdaption}

Um das Arbeiten mit Grammatiken einfacher und übersichtlicher zu
machen, wurde die Darstellung der Nichtterminalzeichen und
Terminalzeichen unterschiedlich gewählt. So wird das Startsymbol in einer
anderen Farbe, als die anderen Nichtterminalzeichen dargestellt wird. Der
Benutzer ist somit in der Lage, in den Produktionen zu erkennen, ob ein
Nichtterminalzeichen das Startsymbol ist oder nicht.\vspace{10pt}

Das Problem wurde so gelöst, dass zum Beispiel dem Scanner der
Nichtterminalzeichen mitgegeben wurde, bei welchem Zeichen es sich um das
Startsymbol handelt. Ein weiteres Beispiel ist die rechte Seite einer
Produktion: Um dem Benutzer auch hier eine Unterscheidung der Symbole leicht zu
machen, muss hier eine farbliche Unterscheidung zwischen Terminalzeichen,
Nichtterminalzeichen und dem Startsymbol erfolgen. Auch dieses Problem wurde auf
die oben genannte Art und Weise gelöst.\vspace{10pt}
