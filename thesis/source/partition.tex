%%
%% $Id$
%%
%% Copyright (c) 2007-2008 Christian Fehler
%% Copyright (c) 2007-2008 Benjamin Mies
%%


\chapter{Aufteilung}\label{Partition}

Im folgenden eine Übersicht, die Auskunft darüber gibt, wer von uns welchen
Teil dieser Diplomarbeit geschrieben hat.


\begin{longtable}{|p{1.30cm}@{}p{7.55cm}@{}p{3.00cm}@{}|}
  \hline
  &
  Einleitung&
  \bm\\
  \hline
  \ref{Grammars}&
  Grammatiken&
  \bm\\
  \hline
  \ref{Machines}& Automaten&
  \cf\\
  \ref{Graph}&
  Graphenansicht&
  \cf\\
  \ref{GraphJGraph}&
  Automatendarstellung mit JGraph&
  \bm\\
  \ref{GraphJGraphAdaptation}&
  Anpassung von JGraph&
  \cf\\
  \ref{Tables}&
  Tabellen&
  \cf\\
  \ref{TablesTransition}&
  Übergangstabelle&
  \cf\\
  \ref{TablesPDA}&
  Keller Operationen Tabelle&
  \bm\\
  \ref{WordNavigation}&
  Wort-Navigation&
  \cf\\
  \ref{WordNavigationDeterministic}&
  Deterministische Navigation&
  \bm\\
  \ref{HistoryPath}&
  Zustands Pfad&
  \cf\\
  \ref{ReachableStates}&
  Erreichbare Zustände&
  \cf\\
  \hline
  \ref{ConverTo}&
  Konvertierung&
  \cf\\
  \ref{ConverToGrammar}&
  Grammatik Konvertierung&
  \bm\\
  \ref{ConverToGrammarRegular}&
  Konvertierung einer regulären Grammatik&
  \bm\\
  \ref{ConverToGrammarContextFree}&
  Umwandeln einer kontextfreien Grammatik&
  \bm\\
  \ref{ConverToMachine}&
  Automaten Konvertierung&
  \cf\\
  \hline
  \ref{Minimize}&
  Minimierung&
  \bm\\
  \hline
  \ref{AutoLayout}&
  Auto Layout&
  \bm\\
  \ref{KerninghanLin}&
  Der Kerninghan-Lin-Algorithmus&
  \bm\\
  \hline
  \ref{Interaction}&
  Interaktion&
  \cf\\
  \ref{InteractionErrorWarning}&
  Fehler und Warnungen&
  \cf\\
  \ref{InteractionGrammar}&
  Grammatik&
  \bm\\
  \ref{InteractionMachine}&
  Automat&
  \cf\\
  \ref{InteractionPDA}&
  Operationen mit dem Automaten Keller&
  \cf\\
  \hline
  \ref{Parser}&
  Parser&
  \cf\\
  \ref{ParserContext}&
  Kontextsensitive Bedingungen&
  \cf\\
  \ref{ParserAdaption}&
  Anpassung der Darstellungsweise&
  \cf\\
  \hline
  \ref{Print}&
  Drucken&
  \bm\\
  \hline
\end{longtable}

\begin{longtable}{|p{1.30cm}@{}p{7.55cm}@{}p{3.00cm}@{}|}
  \hline
  \ref{GUI}&
  Oberfläche&
  \cf\\
  \ref{GUIMain}&
  Gestaltung der Hauptansicht&
  \bm\\
  \ref{GUIRedoUndo}&
  Redo/Undo&
  \bm\\
  \ref{GUIAdaption}&
  Anpassung aller GUI Komponenten&
  \cf\\
  \ref{LookAndFeel}&
  Look \& Feel&
  \cf\\
  \ref{SecondView}&
  Zweite Ansicht&
  \cf\\
  \hline
  \ref{Concepts}&
  Konzepte&
  \cf\\
  \ref{ConceptsLerning}&
  Unterstützung von Lerngruppen&
  \cf\\
  \ref{ConceptsInput}&
  Möglichst wenig Eingabebeschränkungen&
  \cf\\
  \hline
  \ref{Perspective}&
  Perspektiven&
  \cf\\
  \ref{PerspectiveGraphics}&
  Graphische Komponenten&
  \cf\\
  \ref{PerspectiveRegEx}&
  Reguläre Ausdrücke&
  \cf\\
  \ref{PerspectiveLaTeX}&
  \LaTeX-Export&
  \cf\\
  \ref{PerspectiveGrammar}&
  Grammatik Erweiterungen&
  \bm\\
  \ref{PerspectiveLeftrecursion}&
  Linksrekursion&
  \bm\\
  \ref{PerspectiveLeftfactorisation}&
  Linksfaktorisierung&
  \bm\\
  \ref{PerspectiveGrammarWordNavigation}&
  Wortnavigation für Grammatiken&
  \bm\\
  \ref{PerspectiveDetectedWords}&
  Erkannte Wörter ausgeben&
  \cf\\
  \ref{PerspectiveMultiplyWordInput}&
  Eingabe von mehreren Wörtern&
  \cf\\
  \ref{PerspectiveInteraction}&
  Benutzerinteraktion erhöhen&
  \cf\\
  \ref{PerspectiveWordNavigation}&
  Wort-Navigation&
  \cf\\
  \ref{PerspectiveConvertTo}&
  Konvertierung&
  \cf\\
  \ref{PerspectiveReachableStates}&
  Erreichbare Zustände&
  \cf\\
  \ref{PerspectiveMinimize}&
  Minimierung&
  \cf\\
  \ref{PerspectiveAutoLayout}&
  Auto-Layout&
  \bm\\
  \hline
  &
  Fazit&
  \cf\\
  \hline
\end{longtable}
