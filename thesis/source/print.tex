%%
%% $Id$
%%
%% Copyright (c) 2007-2008 Christian Fehler
%% Copyright (c) 2007-2008 Benjamin Mies
%%


\chapter{Drucken}\label{Print}

Da dieses Lernwerkzeug auch beim Lösen und Nachvollziehen der Übungen
Verwendung finden soll, fanden wir es sehr nützlich und wichtig, dass der
Benutzer seine Lösungen auch ausdrucken kann. Sei es um diese bei der
Besprechung der Übung zu vergleichen, oder die Ergebnisse auf dem Papier
nochmal nachzuvollziehen.\vspace{10pt}

Daher sind alle Tabellen, welche bei den verschieden Funktionen des
Lernwerkzeugs Verwendung finden, druckbar. Weiterhin kann auch der konstruierte
Automat gedruckt werden. Zusätzlich können noch alle Zwischenansichten der
Automaten gedruckt werden, sprich die einzelnen Schritte beim Umwandeln oder
Minimieren.\vspace{10pt}

Zusätzlich kann ein konstruierter Automat in das Bildformat png exportiert
werden. So hat man die Möglichkeit einen Automaten in ein PDF oder eine
\LaTeX \ Datei einzubinden.\vspace{10pt}
